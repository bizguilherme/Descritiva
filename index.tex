% Options for packages loaded elsewhere
\PassOptionsToPackage{unicode}{hyperref}
\PassOptionsToPackage{hyphens}{url}
%
\documentclass[
  ignorenonframetext,
]{beamer}
\usepackage{pgfpages}
\setbeamertemplate{caption}[numbered]
\setbeamertemplate{caption label separator}{: }
\setbeamercolor{caption name}{fg=normal text.fg}
\beamertemplatenavigationsymbolsempty
% Prevent slide breaks in the middle of a paragraph
\widowpenalties 1 10000
\raggedbottom
\setbeamertemplate{part page}{
  \centering
  \begin{beamercolorbox}[sep=16pt,center]{part title}
    \usebeamerfont{part title}\insertpart\par
  \end{beamercolorbox}
}
\setbeamertemplate{section page}{
  \centering
  \begin{beamercolorbox}[sep=12pt,center]{section title}
    \usebeamerfont{section title}\insertsection\par
  \end{beamercolorbox}
}
\setbeamertemplate{subsection page}{
  \centering
  \begin{beamercolorbox}[sep=8pt,center]{subsection title}
    \usebeamerfont{subsection title}\insertsubsection\par
  \end{beamercolorbox}
}
\AtBeginPart{
  \frame{\partpage}
}
\AtBeginSection{
  \ifbibliography
  \else
    \frame{\sectionpage}
  \fi
}
\AtBeginSubsection{
  \frame{\subsectionpage}
}

\usepackage{amsmath,amssymb}
\usepackage{iftex}
\ifPDFTeX
  \usepackage[T1]{fontenc}
  \usepackage[utf8]{inputenc}
  \usepackage{textcomp} % provide euro and other symbols
\else % if luatex or xetex
  \usepackage{unicode-math}
  \defaultfontfeatures{Scale=MatchLowercase}
  \defaultfontfeatures[\rmfamily]{Ligatures=TeX,Scale=1}
\fi
\usepackage{lmodern}
\ifPDFTeX\else  
    % xetex/luatex font selection
\fi
% Use upquote if available, for straight quotes in verbatim environments
\IfFileExists{upquote.sty}{\usepackage{upquote}}{}
\IfFileExists{microtype.sty}{% use microtype if available
  \usepackage[]{microtype}
  \UseMicrotypeSet[protrusion]{basicmath} % disable protrusion for tt fonts
}{}
\makeatletter
\@ifundefined{KOMAClassName}{% if non-KOMA class
  \IfFileExists{parskip.sty}{%
    \usepackage{parskip}
  }{% else
    \setlength{\parindent}{0pt}
    \setlength{\parskip}{6pt plus 2pt minus 1pt}}
}{% if KOMA class
  \KOMAoptions{parskip=half}}
\makeatother
\usepackage{xcolor}
\newif\ifbibliography
\setlength{\emergencystretch}{3em} % prevent overfull lines
\setcounter{secnumdepth}{-\maxdimen} % remove section numbering

\usepackage{color}
\usepackage{fancyvrb}
\newcommand{\VerbBar}{|}
\newcommand{\VERB}{\Verb[commandchars=\\\{\}]}
\DefineVerbatimEnvironment{Highlighting}{Verbatim}{commandchars=\\\{\}}
% Add ',fontsize=\small' for more characters per line
\usepackage{framed}
\definecolor{shadecolor}{RGB}{241,243,245}
\newenvironment{Shaded}{\begin{snugshade}}{\end{snugshade}}
\newcommand{\AlertTok}[1]{\textcolor[rgb]{0.68,0.00,0.00}{#1}}
\newcommand{\AnnotationTok}[1]{\textcolor[rgb]{0.37,0.37,0.37}{#1}}
\newcommand{\AttributeTok}[1]{\textcolor[rgb]{0.40,0.45,0.13}{#1}}
\newcommand{\BaseNTok}[1]{\textcolor[rgb]{0.68,0.00,0.00}{#1}}
\newcommand{\BuiltInTok}[1]{\textcolor[rgb]{0.00,0.23,0.31}{#1}}
\newcommand{\CharTok}[1]{\textcolor[rgb]{0.13,0.47,0.30}{#1}}
\newcommand{\CommentTok}[1]{\textcolor[rgb]{0.37,0.37,0.37}{#1}}
\newcommand{\CommentVarTok}[1]{\textcolor[rgb]{0.37,0.37,0.37}{\textit{#1}}}
\newcommand{\ConstantTok}[1]{\textcolor[rgb]{0.56,0.35,0.01}{#1}}
\newcommand{\ControlFlowTok}[1]{\textcolor[rgb]{0.00,0.23,0.31}{\textbf{#1}}}
\newcommand{\DataTypeTok}[1]{\textcolor[rgb]{0.68,0.00,0.00}{#1}}
\newcommand{\DecValTok}[1]{\textcolor[rgb]{0.68,0.00,0.00}{#1}}
\newcommand{\DocumentationTok}[1]{\textcolor[rgb]{0.37,0.37,0.37}{\textit{#1}}}
\newcommand{\ErrorTok}[1]{\textcolor[rgb]{0.68,0.00,0.00}{#1}}
\newcommand{\ExtensionTok}[1]{\textcolor[rgb]{0.00,0.23,0.31}{#1}}
\newcommand{\FloatTok}[1]{\textcolor[rgb]{0.68,0.00,0.00}{#1}}
\newcommand{\FunctionTok}[1]{\textcolor[rgb]{0.28,0.35,0.67}{#1}}
\newcommand{\ImportTok}[1]{\textcolor[rgb]{0.00,0.46,0.62}{#1}}
\newcommand{\InformationTok}[1]{\textcolor[rgb]{0.37,0.37,0.37}{#1}}
\newcommand{\KeywordTok}[1]{\textcolor[rgb]{0.00,0.23,0.31}{\textbf{#1}}}
\newcommand{\NormalTok}[1]{\textcolor[rgb]{0.00,0.23,0.31}{#1}}
\newcommand{\OperatorTok}[1]{\textcolor[rgb]{0.37,0.37,0.37}{#1}}
\newcommand{\OtherTok}[1]{\textcolor[rgb]{0.00,0.23,0.31}{#1}}
\newcommand{\PreprocessorTok}[1]{\textcolor[rgb]{0.68,0.00,0.00}{#1}}
\newcommand{\RegionMarkerTok}[1]{\textcolor[rgb]{0.00,0.23,0.31}{#1}}
\newcommand{\SpecialCharTok}[1]{\textcolor[rgb]{0.37,0.37,0.37}{#1}}
\newcommand{\SpecialStringTok}[1]{\textcolor[rgb]{0.13,0.47,0.30}{#1}}
\newcommand{\StringTok}[1]{\textcolor[rgb]{0.13,0.47,0.30}{#1}}
\newcommand{\VariableTok}[1]{\textcolor[rgb]{0.07,0.07,0.07}{#1}}
\newcommand{\VerbatimStringTok}[1]{\textcolor[rgb]{0.13,0.47,0.30}{#1}}
\newcommand{\WarningTok}[1]{\textcolor[rgb]{0.37,0.37,0.37}{\textit{#1}}}

\providecommand{\tightlist}{%
  \setlength{\itemsep}{0pt}\setlength{\parskip}{0pt}}\usepackage{longtable,booktabs,array}
\usepackage{calc} % for calculating minipage widths
\usepackage{caption}
% Make caption package work with longtable
\makeatletter
\def\fnum@table{\tablename~\thetable}
\makeatother
\usepackage{graphicx}
\makeatletter
\newsavebox\pandoc@box
\newcommand*\pandocbounded[1]{% scales image to fit in text height/width
  \sbox\pandoc@box{#1}%
  \Gscale@div\@tempa{\textheight}{\dimexpr\ht\pandoc@box+\dp\pandoc@box\relax}%
  \Gscale@div\@tempb{\linewidth}{\wd\pandoc@box}%
  \ifdim\@tempb\p@<\@tempa\p@\let\@tempa\@tempb\fi% select the smaller of both
  \ifdim\@tempa\p@<\p@\scalebox{\@tempa}{\usebox\pandoc@box}%
  \else\usebox{\pandoc@box}%
  \fi%
}
% Set default figure placement to htbp
\def\fps@figure{htbp}
\makeatother

\usepackage{array}
\usepackage{caption}
\usepackage{graphicx}
\usepackage{siunitx}
\usepackage[normalem]{ulem}
\usepackage{colortbl}
\usepackage{multirow}
\usepackage{hhline}
\usepackage{calc}
\usepackage{tabularx}
\usepackage{threeparttable}
\usepackage{wrapfig}
\usepackage{adjustbox}
\usepackage{hyperref}
\makeatletter
\@ifpackageloaded{caption}{}{\usepackage{caption}}
\AtBeginDocument{%
\ifdefined\contentsname
  \renewcommand*\contentsname{Table of contents}
\else
  \newcommand\contentsname{Table of contents}
\fi
\ifdefined\listfigurename
  \renewcommand*\listfigurename{List of Figures}
\else
  \newcommand\listfigurename{List of Figures}
\fi
\ifdefined\listtablename
  \renewcommand*\listtablename{List of Tables}
\else
  \newcommand\listtablename{List of Tables}
\fi
\ifdefined\figurename
  \renewcommand*\figurename{Figure}
\else
  \newcommand\figurename{Figure}
\fi
\ifdefined\tablename
  \renewcommand*\tablename{Table}
\else
  \newcommand\tablename{Table}
\fi
}
\@ifpackageloaded{float}{}{\usepackage{float}}
\floatstyle{ruled}
\@ifundefined{c@chapter}{\newfloat{codelisting}{h}{lop}}{\newfloat{codelisting}{h}{lop}[chapter]}
\floatname{codelisting}{Listing}
\newcommand*\listoflistings{\listof{codelisting}{List of Listings}}
\makeatother
\makeatletter
\makeatother
\makeatletter
\@ifpackageloaded{caption}{}{\usepackage{caption}}
\@ifpackageloaded{subcaption}{}{\usepackage{subcaption}}
\makeatother

\usepackage{bookmark}

\IfFileExists{xurl.sty}{\usepackage{xurl}}{} % add URL line breaks if available
\urlstyle{same} % disable monospaced font for URLs
\hypersetup{
  pdftitle={Resumo de Dados},
  pdfauthor={Guilherme Biz},
  hidelinks,
  pdfcreator={LaTeX via pandoc}}


\title{Resumo de Dados}
\author{Guilherme Biz}
\date{}

\begin{document}
\frame{\titlepage}


\begin{frame}{Áreas da Estatística}
\phantomsection\label{uxe1reas-da-estatuxedstica}
\textbf{Descritiva}: é a área da estatística que aplica várias técnicas
para descrever e resumir um conjunto de dados. Os elementos básicos para
esta análise são: tabelas, gráficos e medidas numéricas.

\textbf{Indutiva}: cuida da análise e interpretação dos dados. Tem como
objetivo tirar conclusões sobre a população com base nos resultados
observados em amostras extraídas dessa população.
\end{frame}

\begin{frame}{Variáveis e Dados}
\phantomsection\label{variuxe1veis-e-dados}
\textbf{Variável}: é um atributo, mensurável ou não, sujeito à variação
quantitativa ou qualitativa, no interior de uma população ou amostra.

\textbf{Dados}: são informações inerentes às variáveis que caracterizam
os elementos que constituem a população ou a amostra em estudo.
\end{frame}

\begin{frame}{Tipos de variáveis}
\phantomsection\label{tipos-de-variuxe1veis}
\begin{itemize}
\item
  Qualitativa

  \begin{itemize}
  \item
    Nominal - {Gênero, estado civil, tipo de grão (dentado,semidentado e
    semiduro)\ldots{}}
  \item
    Ordinal - {Grau de escolaridade, resistência à ferrugem
    (r\(>\)mr\(>\)ms\(>\)s)}\footnote<.->{r=resistente; mr=moderadamente
      resistente; ms=moderadamente susceptível; s=susceptível}{\ldots{}}
  \end{itemize}
\item
  Quantitativa

  \begin{itemize}
  \item
    Discreta - {Nº de filhos, nº de grãos por vagem e nº de vagens por
    planta\ldots{}}
  \item
    Contínua - {Peso, Altura da planta e altura da espiga\ldots{}}
  \end{itemize}
\end{itemize}
\end{frame}

\begin{frame}{Conjunto de dados}
\phantomsection\label{conjunto-de-dados}
\end{frame}

\begin{frame}[fragile]{Importando os dados no software R}
\phantomsection\label{importando-os-dados-no-software-r}
\begin{Shaded}
\begin{Highlighting}[]
\NormalTok{dados}\OtherTok{=}\FunctionTok{read.table}\NormalTok{(}\StringTok{"https://www.uel.br/pessoal/biz/pages/arquivos/Agronomia/dados/dados\_milho.txt"}\NormalTok{,}\AttributeTok{head=}\NormalTok{T)}
\FunctionTok{head}\NormalTok{(dados)}
\end{Highlighting}
\end{Shaded}

\begin{verbatim}
  Hibrido Rendimento Ciclo Altura_planta Altura_espiga        Grao Resistencia
1       1       6388    65           242           103     dentado           r
2       2       6166    65           258           134 semidentado           r
3       3       6047    65           240           104 semidentado           s
4       4       5889    66           243           108 semidentado           s
5       5       5823    69           257           128     dentado          ms
6       6       5513    68           241           108 semidentado           s
\end{verbatim}
\end{frame}

\begin{frame}
\begin{itemize}
\item
  \textbf{Dados Brutos}: são dados obtidos diretamente da pesquisa, sem
  terem passados por nenhum processo de síntese ou análise.
\item
  \textbf{Rol:} são os dados colocados em ordem crescente.
\end{itemize}
\end{frame}

\begin{frame}{Dados Brutos}
\phantomsection\label{dados-brutos}
 
  \providecommand{\huxb}[2]{\arrayrulecolor[RGB]{#1}\global\arrayrulewidth=#2pt}
  \providecommand{\huxvb}[2]{\color[RGB]{#1}\vrule width #2pt}
  \providecommand{\huxtpad}[1]{\rule{0pt}{#1}}
  \providecommand{\huxbpad}[1]{\rule[-#1]{0pt}{#1}}

\begin{table}[ht]
\begin{centerbox}
\begin{threeparttable}
\captionsetup{justification=centering,singlelinecheck=off}
\caption{Alturas de espigas (cm) de híbridos de milho para a região de Chapecó, SC - safra: 1987/1988}\setlength{\tabcolsep}{0pt}
\begin{tabular}{l l l l l l l l}


\hhline{>{\huxb{0, 0, 0}{0.4}}->{\huxb{0, 0, 0}{0.4}}->{\huxb{0, 0, 0}{0.4}}->{\huxb{0, 0, 0}{0.4}}->{\huxb{0, 0, 0}{0.4}}->{\huxb{0, 0, 0}{0.4}}->{\huxb{0, 0, 0}{0.4}}->{\huxb{0, 0, 0}{0.4}}-}
\arrayrulecolor{black}

\multicolumn{1}{!{\huxvb{0, 0, 0}{0}}r!{\huxvb{0, 0, 0}{0}}}{\huxtpad{6pt + 1em}\raggedleft \hspace{6pt} 103 \hspace{6pt}\huxbpad{6pt}} &
\multicolumn{1}{r!{\huxvb{0, 0, 0}{0}}}{\huxtpad{6pt + 1em}\raggedleft \hspace{6pt} 128 \hspace{6pt}\huxbpad{6pt}} &
\multicolumn{1}{r!{\huxvb{0, 0, 0}{0}}}{\huxtpad{6pt + 1em}\raggedleft \hspace{6pt} 123 \hspace{6pt}\huxbpad{6pt}} &
\multicolumn{1}{r!{\huxvb{0, 0, 0}{0}}}{\huxtpad{6pt + 1em}\raggedleft \hspace{6pt} 110 \hspace{6pt}\huxbpad{6pt}} &
\multicolumn{1}{r!{\huxvb{0, 0, 0}{0}}}{\huxtpad{6pt + 1em}\raggedleft \hspace{6pt} 134 \hspace{6pt}\huxbpad{6pt}} &
\multicolumn{1}{r!{\huxvb{0, 0, 0}{0}}}{\huxtpad{6pt + 1em}\raggedleft \hspace{6pt} 110 \hspace{6pt}\huxbpad{6pt}} &
\multicolumn{1}{r!{\huxvb{0, 0, 0}{0}}}{\huxtpad{6pt + 1em}\raggedleft \hspace{6pt} 104 \hspace{6pt}\huxbpad{6pt}} &
\multicolumn{1}{r!{\huxvb{0, 0, 0}{0}}}{\huxtpad{6pt + 1em}\raggedleft \hspace{6pt} 120 \hspace{6pt}\huxbpad{6pt}} \tabularnewline[-0.5pt]


\hhline{}
\arrayrulecolor{black}

\multicolumn{1}{!{\huxvb{0, 0, 0}{0}}r!{\huxvb{0, 0, 0}{0}}}{\huxtpad{6pt + 1em}\raggedleft \hspace{6pt} 134 \hspace{6pt}\huxbpad{6pt}} &
\multicolumn{1}{r!{\huxvb{0, 0, 0}{0}}}{\huxtpad{6pt + 1em}\raggedleft \hspace{6pt} 108 \hspace{6pt}\huxbpad{6pt}} &
\multicolumn{1}{r!{\huxvb{0, 0, 0}{0}}}{\huxtpad{6pt + 1em}\raggedleft \hspace{6pt} 117 \hspace{6pt}\huxbpad{6pt}} &
\multicolumn{1}{r!{\huxvb{0, 0, 0}{0}}}{\huxtpad{6pt + 1em}\raggedleft \hspace{6pt} 138 \hspace{6pt}\huxbpad{6pt}} &
\multicolumn{1}{r!{\huxvb{0, 0, 0}{0}}}{\huxtpad{6pt + 1em}\raggedleft \hspace{6pt} 140 \hspace{6pt}\huxbpad{6pt}} &
\multicolumn{1}{r!{\huxvb{0, 0, 0}{0}}}{\huxtpad{6pt + 1em}\raggedleft \hspace{6pt} 128 \hspace{6pt}\huxbpad{6pt}} &
\multicolumn{1}{r!{\huxvb{0, 0, 0}{0}}}{\huxtpad{6pt + 1em}\raggedleft \hspace{6pt} 109 \hspace{6pt}\huxbpad{6pt}} &
\multicolumn{1}{r!{\huxvb{0, 0, 0}{0}}}{\huxtpad{6pt + 1em}\raggedleft \hspace{6pt} 118 \hspace{6pt}\huxbpad{6pt}} \tabularnewline[-0.5pt]


\hhline{}
\arrayrulecolor{black}

\multicolumn{1}{!{\huxvb{0, 0, 0}{0}}r!{\huxvb{0, 0, 0}{0}}}{\huxtpad{6pt + 1em}\raggedleft \hspace{6pt} 104 \hspace{6pt}\huxbpad{6pt}} &
\multicolumn{1}{r!{\huxvb{0, 0, 0}{0}}}{\huxtpad{6pt + 1em}\raggedleft \hspace{6pt} 108 \hspace{6pt}\huxbpad{6pt}} &
\multicolumn{1}{r!{\huxvb{0, 0, 0}{0}}}{\huxtpad{6pt + 1em}\raggedleft \hspace{6pt} 114 \hspace{6pt}\huxbpad{6pt}} &
\multicolumn{1}{r!{\huxvb{0, 0, 0}{0}}}{\huxtpad{6pt + 1em}\raggedleft \hspace{6pt} 149 \hspace{6pt}\huxbpad{6pt}} &
\multicolumn{1}{r!{\huxvb{0, 0, 0}{0}}}{\huxtpad{6pt + 1em}\raggedleft \hspace{6pt} 140 \hspace{6pt}\huxbpad{6pt}} &
\multicolumn{1}{r!{\huxvb{0, 0, 0}{0}}}{\huxtpad{6pt + 1em}\raggedleft \hspace{6pt} 124 \hspace{6pt}\huxbpad{6pt}} &
\multicolumn{1}{r!{\huxvb{0, 0, 0}{0}}}{\huxtpad{6pt + 1em}\raggedleft \hspace{6pt} 137 \hspace{6pt}\huxbpad{6pt}} &
\multicolumn{1}{r!{\huxvb{0, 0, 0}{0}}}{\huxtpad{6pt + 1em}\raggedleft \hspace{6pt} 119 \hspace{6pt}\huxbpad{6pt}} \tabularnewline[-0.5pt]


\hhline{}
\arrayrulecolor{black}

\multicolumn{1}{!{\huxvb{0, 0, 0}{0}}r!{\huxvb{0, 0, 0}{0}}}{\huxtpad{6pt + 1em}\raggedleft \hspace{6pt} 108 \hspace{6pt}\huxbpad{6pt}} &
\multicolumn{1}{r!{\huxvb{0, 0, 0}{0}}}{\huxtpad{6pt + 1em}\raggedleft \hspace{6pt} 103 \hspace{6pt}\huxbpad{6pt}} &
\multicolumn{1}{r!{\huxvb{0, 0, 0}{0}}}{\huxtpad{6pt + 1em}\raggedleft \hspace{6pt} 111 \hspace{6pt}\huxbpad{6pt}} &
\multicolumn{1}{r!{\huxvb{0, 0, 0}{0}}}{\huxtpad{6pt + 1em}\raggedleft \hspace{6pt} 151 \hspace{6pt}\huxbpad{6pt}} &
\multicolumn{1}{r!{\huxvb{0, 0, 0}{0}}}{\huxtpad{6pt + 1em}\raggedleft \hspace{6pt} 139 \hspace{6pt}\huxbpad{6pt}} &
\multicolumn{1}{r!{\huxvb{0, 0, 0}{0}}}{\huxtpad{6pt + 1em}\raggedleft \hspace{6pt} 129 \hspace{6pt}\huxbpad{6pt}} &
\multicolumn{1}{r!{\huxvb{0, 0, 0}{0}}}{\huxtpad{6pt + 1em}\raggedleft \hspace{6pt} 141 \hspace{6pt}\huxbpad{6pt}} &
\multicolumn{1}{r!{\huxvb{0, 0, 0}{0}}}{\huxtpad{6pt + 1em}\raggedleft \hspace{6pt} 138 \hspace{6pt}\huxbpad{6pt}} \tabularnewline[-0.5pt]


\hhline{>{\huxb{0, 0, 0}{0.4}}->{\huxb{0, 0, 0}{0.4}}->{\huxb{0, 0, 0}{0.4}}->{\huxb{0, 0, 0}{0.4}}->{\huxb{0, 0, 0}{0.4}}->{\huxb{0, 0, 0}{0.4}}->{\huxb{0, 0, 0}{0.4}}->{\huxb{0, 0, 0}{0.4}}-}
\arrayrulecolor{black}
\end{tabular}
\end{threeparttable}\par\end{centerbox}

\end{table}
 

 
  \providecommand{\huxb}[2]{\arrayrulecolor[RGB]{#1}\global\arrayrulewidth=#2pt}
  \providecommand{\huxvb}[2]{\color[RGB]{#1}\vrule width #2pt}
  \providecommand{\huxtpad}[1]{\rule{0pt}{#1}}
  \providecommand{\huxbpad}[1]{\rule[-#1]{0pt}{#1}}

\begin{table}[ht]
\begin{centerbox}
\begin{threeparttable}
\setlength{\tabcolsep}{0pt}
\begin{tabular}{l l l l l l l l}


\hhline{>{\huxb{0, 0, 0}{0.4}}->{\huxb{0, 0, 0}{0.4}}->{\huxb{0, 0, 0}{0.4}}->{\huxb{0, 0, 0}{0.4}}->{\huxb{0, 0, 0}{0.4}}->{\huxb{0, 0, 0}{0.4}}->{\huxb{0, 0, 0}{0.4}}->{\huxb{0, 0, 0}{0.4}}-}
\arrayrulecolor{black}

\multicolumn{1}{!{\huxvb{0, 0, 0}{0}}r!{\huxvb{0, 0, 0}{0}}}{\huxtpad{6pt + 1em}\raggedleft \hspace{6pt} 103 \hspace{6pt}\huxbpad{6pt}} &
\multicolumn{1}{r!{\huxvb{0, 0, 0}{0}}}{\huxtpad{6pt + 1em}\raggedleft \hspace{6pt} 128 \hspace{6pt}\huxbpad{6pt}} &
\multicolumn{1}{r!{\huxvb{0, 0, 0}{0}}}{\huxtpad{6pt + 1em}\raggedleft \hspace{6pt} 123 \hspace{6pt}\huxbpad{6pt}} &
\multicolumn{1}{r!{\huxvb{0, 0, 0}{0}}}{\huxtpad{6pt + 1em}\raggedleft \hspace{6pt} 110 \hspace{6pt}\huxbpad{6pt}} &
\multicolumn{1}{r!{\huxvb{0, 0, 0}{0}}}{\huxtpad{6pt + 1em}\raggedleft \hspace{6pt} 134 \hspace{6pt}\huxbpad{6pt}} &
\multicolumn{1}{r!{\huxvb{0, 0, 0}{0}}}{\huxtpad{6pt + 1em}\raggedleft \hspace{6pt} 110 \hspace{6pt}\huxbpad{6pt}} &
\multicolumn{1}{r!{\huxvb{0, 0, 0}{0}}}{\huxtpad{6pt + 1em}\raggedleft \hspace{6pt} 104 \hspace{6pt}\huxbpad{6pt}} &
\multicolumn{1}{r!{\huxvb{0, 0, 0}{0}}}{\huxtpad{6pt + 1em}\raggedleft \hspace{6pt} 120 \hspace{6pt}\huxbpad{6pt}} \tabularnewline[-0.5pt]


\hhline{}
\arrayrulecolor{black}

\multicolumn{1}{!{\huxvb{0, 0, 0}{0}}r!{\huxvb{0, 0, 0}{0}}}{\huxtpad{6pt + 1em}\raggedleft \hspace{6pt} 134 \hspace{6pt}\huxbpad{6pt}} &
\multicolumn{1}{r!{\huxvb{0, 0, 0}{0}}}{\huxtpad{6pt + 1em}\raggedleft \hspace{6pt} 108 \hspace{6pt}\huxbpad{6pt}} &
\multicolumn{1}{r!{\huxvb{0, 0, 0}{0}}}{\huxtpad{6pt + 1em}\raggedleft \hspace{6pt} 117 \hspace{6pt}\huxbpad{6pt}} &
\multicolumn{1}{r!{\huxvb{0, 0, 0}{0}}}{\huxtpad{6pt + 1em}\raggedleft \hspace{6pt} 138 \hspace{6pt}\huxbpad{6pt}} &
\multicolumn{1}{r!{\huxvb{0, 0, 0}{0}}}{\huxtpad{6pt + 1em}\raggedleft \hspace{6pt} 140 \hspace{6pt}\huxbpad{6pt}} &
\multicolumn{1}{r!{\huxvb{0, 0, 0}{0}}}{\huxtpad{6pt + 1em}\raggedleft \hspace{6pt} 128 \hspace{6pt}\huxbpad{6pt}} &
\multicolumn{1}{r!{\huxvb{0, 0, 0}{0}}}{\huxtpad{6pt + 1em}\raggedleft \hspace{6pt} 109 \hspace{6pt}\huxbpad{6pt}} &
\multicolumn{1}{r!{\huxvb{0, 0, 0}{0}}}{\huxtpad{6pt + 1em}\raggedleft \hspace{6pt} 118 \hspace{6pt}\huxbpad{6pt}} \tabularnewline[-0.5pt]


\hhline{}
\arrayrulecolor{black}

\multicolumn{1}{!{\huxvb{0, 0, 0}{0}}r!{\huxvb{0, 0, 0}{0}}}{\huxtpad{6pt + 1em}\raggedleft \hspace{6pt} 104 \hspace{6pt}\huxbpad{6pt}} &
\multicolumn{1}{r!{\huxvb{0, 0, 0}{0}}}{\huxtpad{6pt + 1em}\raggedleft \hspace{6pt} 108 \hspace{6pt}\huxbpad{6pt}} &
\multicolumn{1}{r!{\huxvb{0, 0, 0}{0}}}{\huxtpad{6pt + 1em}\raggedleft \hspace{6pt} 114 \hspace{6pt}\huxbpad{6pt}} &
\multicolumn{1}{r!{\huxvb{0, 0, 0}{0}}}{\huxtpad{6pt + 1em}\raggedleft \hspace{6pt} 149 \hspace{6pt}\huxbpad{6pt}} &
\multicolumn{1}{r!{\huxvb{0, 0, 0}{0}}}{\huxtpad{6pt + 1em}\raggedleft \hspace{6pt} 140 \hspace{6pt}\huxbpad{6pt}} &
\multicolumn{1}{r!{\huxvb{0, 0, 0}{0}}}{\huxtpad{6pt + 1em}\raggedleft \hspace{6pt} 124 \hspace{6pt}\huxbpad{6pt}} &
\multicolumn{1}{r!{\huxvb{0, 0, 0}{0}}}{\huxtpad{6pt + 1em}\raggedleft \hspace{6pt} 137 \hspace{6pt}\huxbpad{6pt}} &
\multicolumn{1}{r!{\huxvb{0, 0, 0}{0}}}{\huxtpad{6pt + 1em}\raggedleft \hspace{6pt} 119 \hspace{6pt}\huxbpad{6pt}} \tabularnewline[-0.5pt]


\hhline{}
\arrayrulecolor{black}

\multicolumn{1}{!{\huxvb{0, 0, 0}{0}}r!{\huxvb{0, 0, 0}{0}}}{\huxtpad{6pt + 1em}\raggedleft \hspace{6pt} 108 \hspace{6pt}\huxbpad{6pt}} &
\multicolumn{1}{r!{\huxvb{0, 0, 0}{0}}}{\huxtpad{6pt + 1em}\raggedleft \hspace{6pt} 103 \hspace{6pt}\huxbpad{6pt}} &
\multicolumn{1}{r!{\huxvb{0, 0, 0}{0}}}{\huxtpad{6pt + 1em}\raggedleft \hspace{6pt} 111 \hspace{6pt}\huxbpad{6pt}} &
\multicolumn{1}{r!{\huxvb{0, 0, 0}{0}}}{\huxtpad{6pt + 1em}\raggedleft \hspace{6pt} 151 \hspace{6pt}\huxbpad{6pt}} &
\multicolumn{1}{r!{\huxvb{0, 0, 0}{0}}}{\huxtpad{6pt + 1em}\raggedleft \hspace{6pt} 139 \hspace{6pt}\huxbpad{6pt}} &
\multicolumn{1}{r!{\huxvb{0, 0, 0}{0}}}{\huxtpad{6pt + 1em}\raggedleft \hspace{6pt} 129 \hspace{6pt}\huxbpad{6pt}} &
\multicolumn{1}{r!{\huxvb{0, 0, 0}{0}}}{\huxtpad{6pt + 1em}\raggedleft \hspace{6pt} 141 \hspace{6pt}\huxbpad{6pt}} &
\multicolumn{1}{r!{\huxvb{0, 0, 0}{0}}}{\huxtpad{6pt + 1em}\raggedleft \hspace{6pt} 138 \hspace{6pt}\huxbpad{6pt}} \tabularnewline[-0.5pt]


\hhline{>{\huxb{0, 0, 0}{0.4}}->{\huxb{0, 0, 0}{0.4}}->{\huxb{0, 0, 0}{0.4}}->{\huxb{0, 0, 0}{0.4}}->{\huxb{0, 0, 0}{0.4}}->{\huxb{0, 0, 0}{0.4}}->{\huxb{0, 0, 0}{0.4}}->{\huxb{0, 0, 0}{0.4}}-}
\arrayrulecolor{black}
\end{tabular}
\end{threeparttable}\par\end{centerbox}

\end{table}
 
\end{frame}

\begin{frame}{Rol}
\phantomsection\label{rol}
 
  \providecommand{\huxb}[2]{\arrayrulecolor[RGB]{#1}\global\arrayrulewidth=#2pt}
  \providecommand{\huxvb}[2]{\color[RGB]{#1}\vrule width #2pt}
  \providecommand{\huxtpad}[1]{\rule{0pt}{#1}}
  \providecommand{\huxbpad}[1]{\rule[-#1]{0pt}{#1}}

\begin{table}[ht]
\begin{centerbox}
\begin{threeparttable}
\captionsetup{justification=centering,singlelinecheck=off}
\caption{Alturas de espigas (cm) de híbridos de milho para a região de Chapecó, SC - safra: 1987/1988}\setlength{\tabcolsep}{0pt}
\begin{tabular}{l l l l l l l l}


\hhline{>{\huxb{0, 0, 0}{0.4}}->{\huxb{0, 0, 0}{0.4}}->{\huxb{0, 0, 0}{0.4}}->{\huxb{0, 0, 0}{0.4}}->{\huxb{0, 0, 0}{0.4}}->{\huxb{0, 0, 0}{0.4}}->{\huxb{0, 0, 0}{0.4}}->{\huxb{0, 0, 0}{0.4}}-}
\arrayrulecolor{black}

\multicolumn{1}{!{\huxvb{0, 0, 0}{0}}r!{\huxvb{0, 0, 0}{0}}}{\huxtpad{6pt + 1em}\raggedleft \hspace{6pt} 103 \hspace{6pt}\huxbpad{6pt}} &
\multicolumn{1}{r!{\huxvb{0, 0, 0}{0}}}{\huxtpad{6pt + 1em}\raggedleft \hspace{6pt} 103 \hspace{6pt}\huxbpad{6pt}} &
\multicolumn{1}{r!{\huxvb{0, 0, 0}{0}}}{\huxtpad{6pt + 1em}\raggedleft \hspace{6pt} 104 \hspace{6pt}\huxbpad{6pt}} &
\multicolumn{1}{r!{\huxvb{0, 0, 0}{0}}}{\huxtpad{6pt + 1em}\raggedleft \hspace{6pt} 104 \hspace{6pt}\huxbpad{6pt}} &
\multicolumn{1}{r!{\huxvb{0, 0, 0}{0}}}{\huxtpad{6pt + 1em}\raggedleft \hspace{6pt} 108 \hspace{6pt}\huxbpad{6pt}} &
\multicolumn{1}{r!{\huxvb{0, 0, 0}{0}}}{\huxtpad{6pt + 1em}\raggedleft \hspace{6pt} 108 \hspace{6pt}\huxbpad{6pt}} &
\multicolumn{1}{r!{\huxvb{0, 0, 0}{0}}}{\huxtpad{6pt + 1em}\raggedleft \hspace{6pt} 108 \hspace{6pt}\huxbpad{6pt}} &
\multicolumn{1}{r!{\huxvb{0, 0, 0}{0}}}{\huxtpad{6pt + 1em}\raggedleft \hspace{6pt} 109 \hspace{6pt}\huxbpad{6pt}} \tabularnewline[-0.5pt]


\hhline{}
\arrayrulecolor{black}

\multicolumn{1}{!{\huxvb{0, 0, 0}{0}}r!{\huxvb{0, 0, 0}{0}}}{\huxtpad{6pt + 1em}\raggedleft \hspace{6pt} 110 \hspace{6pt}\huxbpad{6pt}} &
\multicolumn{1}{r!{\huxvb{0, 0, 0}{0}}}{\huxtpad{6pt + 1em}\raggedleft \hspace{6pt} 110 \hspace{6pt}\huxbpad{6pt}} &
\multicolumn{1}{r!{\huxvb{0, 0, 0}{0}}}{\huxtpad{6pt + 1em}\raggedleft \hspace{6pt} 111 \hspace{6pt}\huxbpad{6pt}} &
\multicolumn{1}{r!{\huxvb{0, 0, 0}{0}}}{\huxtpad{6pt + 1em}\raggedleft \hspace{6pt} 114 \hspace{6pt}\huxbpad{6pt}} &
\multicolumn{1}{r!{\huxvb{0, 0, 0}{0}}}{\huxtpad{6pt + 1em}\raggedleft \hspace{6pt} 117 \hspace{6pt}\huxbpad{6pt}} &
\multicolumn{1}{r!{\huxvb{0, 0, 0}{0}}}{\huxtpad{6pt + 1em}\raggedleft \hspace{6pt} 118 \hspace{6pt}\huxbpad{6pt}} &
\multicolumn{1}{r!{\huxvb{0, 0, 0}{0}}}{\huxtpad{6pt + 1em}\raggedleft \hspace{6pt} 119 \hspace{6pt}\huxbpad{6pt}} &
\multicolumn{1}{r!{\huxvb{0, 0, 0}{0}}}{\huxtpad{6pt + 1em}\raggedleft \hspace{6pt} 120 \hspace{6pt}\huxbpad{6pt}} \tabularnewline[-0.5pt]


\hhline{}
\arrayrulecolor{black}

\multicolumn{1}{!{\huxvb{0, 0, 0}{0}}r!{\huxvb{0, 0, 0}{0}}}{\huxtpad{6pt + 1em}\raggedleft \hspace{6pt} 123 \hspace{6pt}\huxbpad{6pt}} &
\multicolumn{1}{r!{\huxvb{0, 0, 0}{0}}}{\huxtpad{6pt + 1em}\raggedleft \hspace{6pt} 124 \hspace{6pt}\huxbpad{6pt}} &
\multicolumn{1}{r!{\huxvb{0, 0, 0}{0}}}{\huxtpad{6pt + 1em}\raggedleft \hspace{6pt} 128 \hspace{6pt}\huxbpad{6pt}} &
\multicolumn{1}{r!{\huxvb{0, 0, 0}{0}}}{\huxtpad{6pt + 1em}\raggedleft \hspace{6pt} 128 \hspace{6pt}\huxbpad{6pt}} &
\multicolumn{1}{r!{\huxvb{0, 0, 0}{0}}}{\huxtpad{6pt + 1em}\raggedleft \hspace{6pt} 129 \hspace{6pt}\huxbpad{6pt}} &
\multicolumn{1}{r!{\huxvb{0, 0, 0}{0}}}{\huxtpad{6pt + 1em}\raggedleft \hspace{6pt} 134 \hspace{6pt}\huxbpad{6pt}} &
\multicolumn{1}{r!{\huxvb{0, 0, 0}{0}}}{\huxtpad{6pt + 1em}\raggedleft \hspace{6pt} 134 \hspace{6pt}\huxbpad{6pt}} &
\multicolumn{1}{r!{\huxvb{0, 0, 0}{0}}}{\huxtpad{6pt + 1em}\raggedleft \hspace{6pt} 137 \hspace{6pt}\huxbpad{6pt}} \tabularnewline[-0.5pt]


\hhline{}
\arrayrulecolor{black}

\multicolumn{1}{!{\huxvb{0, 0, 0}{0}}r!{\huxvb{0, 0, 0}{0}}}{\huxtpad{6pt + 1em}\raggedleft \hspace{6pt} 138 \hspace{6pt}\huxbpad{6pt}} &
\multicolumn{1}{r!{\huxvb{0, 0, 0}{0}}}{\huxtpad{6pt + 1em}\raggedleft \hspace{6pt} 138 \hspace{6pt}\huxbpad{6pt}} &
\multicolumn{1}{r!{\huxvb{0, 0, 0}{0}}}{\huxtpad{6pt + 1em}\raggedleft \hspace{6pt} 139 \hspace{6pt}\huxbpad{6pt}} &
\multicolumn{1}{r!{\huxvb{0, 0, 0}{0}}}{\huxtpad{6pt + 1em}\raggedleft \hspace{6pt} 140 \hspace{6pt}\huxbpad{6pt}} &
\multicolumn{1}{r!{\huxvb{0, 0, 0}{0}}}{\huxtpad{6pt + 1em}\raggedleft \hspace{6pt} 140 \hspace{6pt}\huxbpad{6pt}} &
\multicolumn{1}{r!{\huxvb{0, 0, 0}{0}}}{\huxtpad{6pt + 1em}\raggedleft \hspace{6pt} 141 \hspace{6pt}\huxbpad{6pt}} &
\multicolumn{1}{r!{\huxvb{0, 0, 0}{0}}}{\huxtpad{6pt + 1em}\raggedleft \hspace{6pt} 149 \hspace{6pt}\huxbpad{6pt}} &
\multicolumn{1}{r!{\huxvb{0, 0, 0}{0}}}{\huxtpad{6pt + 1em}\raggedleft \hspace{6pt} 151 \hspace{6pt}\huxbpad{6pt}} \tabularnewline[-0.5pt]


\hhline{>{\huxb{0, 0, 0}{0.4}}->{\huxb{0, 0, 0}{0.4}}->{\huxb{0, 0, 0}{0.4}}->{\huxb{0, 0, 0}{0.4}}->{\huxb{0, 0, 0}{0.4}}->{\huxb{0, 0, 0}{0.4}}->{\huxb{0, 0, 0}{0.4}}->{\huxb{0, 0, 0}{0.4}}-}
\arrayrulecolor{black}
\end{tabular}
\end{threeparttable}\par\end{centerbox}

\end{table}
 

 
  \providecommand{\huxb}[2]{\arrayrulecolor[RGB]{#1}\global\arrayrulewidth=#2pt}
  \providecommand{\huxvb}[2]{\color[RGB]{#1}\vrule width #2pt}
  \providecommand{\huxtpad}[1]{\rule{0pt}{#1}}
  \providecommand{\huxbpad}[1]{\rule[-#1]{0pt}{#1}}

\begin{table}[ht]
\begin{centerbox}
\begin{threeparttable}
\setlength{\tabcolsep}{0pt}
\begin{tabular}{l l l l l l l l}


\hhline{>{\huxb{0, 0, 0}{0.4}}->{\huxb{0, 0, 0}{0.4}}->{\huxb{0, 0, 0}{0.4}}->{\huxb{0, 0, 0}{0.4}}->{\huxb{0, 0, 0}{0.4}}->{\huxb{0, 0, 0}{0.4}}->{\huxb{0, 0, 0}{0.4}}->{\huxb{0, 0, 0}{0.4}}-}
\arrayrulecolor{black}

\multicolumn{1}{!{\huxvb{0, 0, 0}{0}}r!{\huxvb{0, 0, 0}{0}}}{\huxtpad{6pt + 1em}\raggedleft \hspace{6pt} 103 \hspace{6pt}\huxbpad{6pt}} &
\multicolumn{1}{r!{\huxvb{0, 0, 0}{0}}}{\huxtpad{6pt + 1em}\raggedleft \hspace{6pt} 103 \hspace{6pt}\huxbpad{6pt}} &
\multicolumn{1}{r!{\huxvb{0, 0, 0}{0}}}{\huxtpad{6pt + 1em}\raggedleft \hspace{6pt} 104 \hspace{6pt}\huxbpad{6pt}} &
\multicolumn{1}{r!{\huxvb{0, 0, 0}{0}}}{\huxtpad{6pt + 1em}\raggedleft \hspace{6pt} 104 \hspace{6pt}\huxbpad{6pt}} &
\multicolumn{1}{r!{\huxvb{0, 0, 0}{0}}}{\huxtpad{6pt + 1em}\raggedleft \hspace{6pt} 108 \hspace{6pt}\huxbpad{6pt}} &
\multicolumn{1}{r!{\huxvb{0, 0, 0}{0}}}{\huxtpad{6pt + 1em}\raggedleft \hspace{6pt} 108 \hspace{6pt}\huxbpad{6pt}} &
\multicolumn{1}{r!{\huxvb{0, 0, 0}{0}}}{\huxtpad{6pt + 1em}\raggedleft \hspace{6pt} 108 \hspace{6pt}\huxbpad{6pt}} &
\multicolumn{1}{r!{\huxvb{0, 0, 0}{0}}}{\huxtpad{6pt + 1em}\raggedleft \hspace{6pt} 109 \hspace{6pt}\huxbpad{6pt}} \tabularnewline[-0.5pt]


\hhline{}
\arrayrulecolor{black}

\multicolumn{1}{!{\huxvb{0, 0, 0}{0}}r!{\huxvb{0, 0, 0}{0}}}{\huxtpad{6pt + 1em}\raggedleft \hspace{6pt} 110 \hspace{6pt}\huxbpad{6pt}} &
\multicolumn{1}{r!{\huxvb{0, 0, 0}{0}}}{\huxtpad{6pt + 1em}\raggedleft \hspace{6pt} 110 \hspace{6pt}\huxbpad{6pt}} &
\multicolumn{1}{r!{\huxvb{0, 0, 0}{0}}}{\huxtpad{6pt + 1em}\raggedleft \hspace{6pt} 111 \hspace{6pt}\huxbpad{6pt}} &
\multicolumn{1}{r!{\huxvb{0, 0, 0}{0}}}{\huxtpad{6pt + 1em}\raggedleft \hspace{6pt} 114 \hspace{6pt}\huxbpad{6pt}} &
\multicolumn{1}{r!{\huxvb{0, 0, 0}{0}}}{\huxtpad{6pt + 1em}\raggedleft \hspace{6pt} 117 \hspace{6pt}\huxbpad{6pt}} &
\multicolumn{1}{r!{\huxvb{0, 0, 0}{0}}}{\huxtpad{6pt + 1em}\raggedleft \hspace{6pt} 118 \hspace{6pt}\huxbpad{6pt}} &
\multicolumn{1}{r!{\huxvb{0, 0, 0}{0}}}{\huxtpad{6pt + 1em}\raggedleft \hspace{6pt} 119 \hspace{6pt}\huxbpad{6pt}} &
\multicolumn{1}{r!{\huxvb{0, 0, 0}{0}}}{\huxtpad{6pt + 1em}\raggedleft \hspace{6pt} 120 \hspace{6pt}\huxbpad{6pt}} \tabularnewline[-0.5pt]


\hhline{}
\arrayrulecolor{black}

\multicolumn{1}{!{\huxvb{0, 0, 0}{0}}r!{\huxvb{0, 0, 0}{0}}}{\huxtpad{6pt + 1em}\raggedleft \hspace{6pt} 123 \hspace{6pt}\huxbpad{6pt}} &
\multicolumn{1}{r!{\huxvb{0, 0, 0}{0}}}{\huxtpad{6pt + 1em}\raggedleft \hspace{6pt} 124 \hspace{6pt}\huxbpad{6pt}} &
\multicolumn{1}{r!{\huxvb{0, 0, 0}{0}}}{\huxtpad{6pt + 1em}\raggedleft \hspace{6pt} 128 \hspace{6pt}\huxbpad{6pt}} &
\multicolumn{1}{r!{\huxvb{0, 0, 0}{0}}}{\huxtpad{6pt + 1em}\raggedleft \hspace{6pt} 128 \hspace{6pt}\huxbpad{6pt}} &
\multicolumn{1}{r!{\huxvb{0, 0, 0}{0}}}{\huxtpad{6pt + 1em}\raggedleft \hspace{6pt} 129 \hspace{6pt}\huxbpad{6pt}} &
\multicolumn{1}{r!{\huxvb{0, 0, 0}{0}}}{\huxtpad{6pt + 1em}\raggedleft \hspace{6pt} 134 \hspace{6pt}\huxbpad{6pt}} &
\multicolumn{1}{r!{\huxvb{0, 0, 0}{0}}}{\huxtpad{6pt + 1em}\raggedleft \hspace{6pt} 134 \hspace{6pt}\huxbpad{6pt}} &
\multicolumn{1}{r!{\huxvb{0, 0, 0}{0}}}{\huxtpad{6pt + 1em}\raggedleft \hspace{6pt} 137 \hspace{6pt}\huxbpad{6pt}} \tabularnewline[-0.5pt]


\hhline{}
\arrayrulecolor{black}

\multicolumn{1}{!{\huxvb{0, 0, 0}{0}}r!{\huxvb{0, 0, 0}{0}}}{\huxtpad{6pt + 1em}\raggedleft \hspace{6pt} 138 \hspace{6pt}\huxbpad{6pt}} &
\multicolumn{1}{r!{\huxvb{0, 0, 0}{0}}}{\huxtpad{6pt + 1em}\raggedleft \hspace{6pt} 138 \hspace{6pt}\huxbpad{6pt}} &
\multicolumn{1}{r!{\huxvb{0, 0, 0}{0}}}{\huxtpad{6pt + 1em}\raggedleft \hspace{6pt} 139 \hspace{6pt}\huxbpad{6pt}} &
\multicolumn{1}{r!{\huxvb{0, 0, 0}{0}}}{\huxtpad{6pt + 1em}\raggedleft \hspace{6pt} 140 \hspace{6pt}\huxbpad{6pt}} &
\multicolumn{1}{r!{\huxvb{0, 0, 0}{0}}}{\huxtpad{6pt + 1em}\raggedleft \hspace{6pt} 140 \hspace{6pt}\huxbpad{6pt}} &
\multicolumn{1}{r!{\huxvb{0, 0, 0}{0}}}{\huxtpad{6pt + 1em}\raggedleft \hspace{6pt} 141 \hspace{6pt}\huxbpad{6pt}} &
\multicolumn{1}{r!{\huxvb{0, 0, 0}{0}}}{\huxtpad{6pt + 1em}\raggedleft \hspace{6pt} 149 \hspace{6pt}\huxbpad{6pt}} &
\multicolumn{1}{r!{\huxvb{0, 0, 0}{0}}}{\huxtpad{6pt + 1em}\raggedleft \hspace{6pt} 151 \hspace{6pt}\huxbpad{6pt}} \tabularnewline[-0.5pt]


\hhline{>{\huxb{0, 0, 0}{0.4}}->{\huxb{0, 0, 0}{0.4}}->{\huxb{0, 0, 0}{0.4}}->{\huxb{0, 0, 0}{0.4}}->{\huxb{0, 0, 0}{0.4}}->{\huxb{0, 0, 0}{0.4}}->{\huxb{0, 0, 0}{0.4}}->{\huxb{0, 0, 0}{0.4}}-}
\arrayrulecolor{black}
\end{tabular}
\end{threeparttable}\par\end{centerbox}

\end{table}
 
\end{frame}

\begin{frame}{Representação Tabular}
\phantomsection\label{representauxe7uxe3o-tabular}
\begin{itemize}
\item
  \textbf{Título}: índica a natureza do fato estudado, as variáveis
  escolhidas na análise do fato, o local e época.
\item
  \textbf{Corpo}: é o conjunto de linhas e colunas que contém,
  respectivamente, as séries horizontais e verticais de informações.
\item
  \textbf{Cabeçalho}: designa a natureza do conteúdo de cada coluna.
\item
  \textbf{Coluna indicadora}: mostra natureza do conteúdo de cada linha.
\end{itemize}
\end{frame}

\begin{frame}
\pandocbounded{\includegraphics[keepaspectratio]{index_files/images/tabela.png}}
\end{frame}

\begin{frame}
\begin{block}{Distribuições de frequências de variáveis qualitativas e
representações gráficas}
\phantomsection\label{distribuiuxe7uxf5es-de-frequuxeancias-de-variuxe1veis-qualitativas-e-representauxe7uxf5es-gruxe1ficas}
\begin{itemize}
\item
  Uma tabela de \textbf{distribuição de frequências} é a
  correspondências entre as categorias ou valores de uma variável e as
  respectivas frequências que foram observadas.
\item
  Para esse tipo de variável os gráficos mais utilizados são: colunas,
  barras e de setores.
\end{itemize}
\end{block}
\end{frame}

\begin{frame}[fragile]
Tabela: Distribuição de Frequências da resistência à ferrugem de 32
híbridos de milho recomendados para a região de Chapecó, safra- 1987/88.

\begin{longtable}[]{@{}
  >{\centering\arraybackslash}p{(\linewidth - 8\tabcolsep) * \real{0.2000}}
  >{\centering\arraybackslash}p{(\linewidth - 8\tabcolsep) * \real{0.2000}}
  >{\centering\arraybackslash}p{(\linewidth - 8\tabcolsep) * \real{0.2000}}
  >{\centering\arraybackslash}p{(\linewidth - 8\tabcolsep) * \real{0.2000}}
  >{\centering\arraybackslash}p{(\linewidth - 8\tabcolsep) * \real{0.2000}}@{}}
\toprule\noalign{}
\begin{minipage}[b]{\linewidth}\centering
Resitência à ferrugem
\end{minipage} & \begin{minipage}[b]{\linewidth}\centering
Frequência absoluta\((n_i)\)
\end{minipage} & \begin{minipage}[b]{\linewidth}\centering
Frequência relativa \((f_i)\)
\end{minipage} & \begin{minipage}[b]{\linewidth}\centering
Porcentagem (\%)
\end{minipage} & \begin{minipage}[b]{\linewidth}\centering
Porcentagem acumulada
\end{minipage} \\
\midrule\noalign{}
\endhead
r & 10 & 0,3125 & 31,25 & 31,25 \\
mr & 6 & 0,1875 & 18,75 & 50,00 \\
ms & 9 & 0,2813 & 28,13 & 78,13 \\
s & 7 & 0,2187 & 21,87 & 100,00 \\
Total & 32 & 1 & 100 & \(-\) \\
\bottomrule\noalign{}
\end{longtable}

\begin{Shaded}
\begin{Highlighting}[]
\FunctionTok{require}\NormalTok{(epiDisplay)}
\FunctionTok{tab1}\NormalTok{(}\FunctionTok{factor}\NormalTok{(dados}\SpecialCharTok{$}\NormalTok{Resistencia,}\AttributeTok{levels =} \FunctionTok{c}\NormalTok{(}\StringTok{"r"}\NormalTok{,}\StringTok{"mr"}\NormalTok{,}\StringTok{"ms"}\NormalTok{,}\StringTok{"s"}\NormalTok{)))}
\end{Highlighting}
\end{Shaded}
\end{frame}

\begin{frame}[fragile]
\begin{block}{Gráficos para variáveis qualitativas - Colunas}
\phantomsection\label{gruxe1ficos-para-variuxe1veis-qualitativas---colunas}
\begin{Shaded}
\begin{Highlighting}[]
\FunctionTok{barplot}\NormalTok{(}\FunctionTok{table}\NormalTok{(dados}\SpecialCharTok{$}\NormalTok{Resistencia)[}\FunctionTok{c}\NormalTok{(}\StringTok{"r"}\NormalTok{,}\StringTok{"mr"}\NormalTok{,}\StringTok{"ms"}\NormalTok{,}\StringTok{"s"}\NormalTok{)],}\AttributeTok{xlab=}\StringTok{"Resistência à ferrugem"}\NormalTok{,}\AttributeTok{ylab=}\StringTok{"Frequências"}\NormalTok{,}\AttributeTok{col=}\StringTok{"green4"}\NormalTok{,}\AttributeTok{density=}\DecValTok{50}\NormalTok{,}\AttributeTok{ylim=}\FunctionTok{c}\NormalTok{(}\DecValTok{0}\NormalTok{,}\DecValTok{12}\NormalTok{))}
\FunctionTok{abline}\NormalTok{(}\AttributeTok{h=}\DecValTok{0}\NormalTok{)}
\end{Highlighting}
\end{Shaded}

\begin{center}
\pandocbounded{\includegraphics[keepaspectratio]{index_files/figure-beamer/unnamed-chunk-5-1.pdf}}
\end{center}

\textbf{Figura}: Resistência à ferrugem de híbridos de milho, para a
região de Chapecó, 1987/88
\end{block}
\end{frame}

\begin{frame}[fragile]
\begin{block}{Gráficos para variáveis qualitativas - Barras}
\phantomsection\label{gruxe1ficos-para-variuxe1veis-qualitativas---barras}
\begin{Shaded}
\begin{Highlighting}[]
\FunctionTok{barplot}\NormalTok{(}\FunctionTok{table}\NormalTok{(dados}\SpecialCharTok{$}\NormalTok{Resistencia)[}\FunctionTok{c}\NormalTok{(}\StringTok{"r"}\NormalTok{,}\StringTok{"mr"}\NormalTok{,}\StringTok{"ms"}\NormalTok{,}\StringTok{"s"}\NormalTok{)],}\AttributeTok{xlab=}\StringTok{"Resistência à ferrugem"}\NormalTok{,}\AttributeTok{ylab=}\StringTok{"Frequências"}\NormalTok{,}\AttributeTok{col=}\StringTok{"green4"}\NormalTok{,}\AttributeTok{density=}\DecValTok{50}\NormalTok{,}\AttributeTok{xlim=}\FunctionTok{c}\NormalTok{(}\DecValTok{0}\NormalTok{,}\DecValTok{12}\NormalTok{),}\AttributeTok{horiz =}\NormalTok{ T)}
\FunctionTok{abline}\NormalTok{(}\AttributeTok{v=}\DecValTok{0}\NormalTok{)}
\end{Highlighting}
\end{Shaded}

\begin{center}
\pandocbounded{\includegraphics[keepaspectratio]{index_files/figure-beamer/unnamed-chunk-6-1.pdf}}
\end{center}

\textbf{Figura}: Resistência à ferrugem de híbridos de milho, para a
região de Chapecó, 1987/88
\end{block}
\end{frame}

\begin{frame}[fragile]
\begin{block}{Gráficos para variáveis qualitativas - Setores}
\phantomsection\label{gruxe1ficos-para-variuxe1veis-qualitativas---setores}
\begin{Shaded}
\begin{Highlighting}[]
\NormalTok{resis}\OtherTok{=}\FunctionTok{table}\NormalTok{(dados}\SpecialCharTok{$}\NormalTok{Resistencia)[}\FunctionTok{c}\NormalTok{(}\StringTok{"r"}\NormalTok{,}\StringTok{"mr"}\NormalTok{,}\StringTok{"ms"}\NormalTok{,}\StringTok{"s"}\NormalTok{)]}
\FunctionTok{names}\NormalTok{(resis)}\OtherTok{=}\FunctionTok{paste}\NormalTok{(}\FunctionTok{c}\NormalTok{(}\StringTok{"resistentes"}\NormalTok{,}\StringTok{"mod. resistentes"}\NormalTok{,}
                     \StringTok{"mod. susceptível"}\NormalTok{,}\StringTok{"susceptível"}\NormalTok{),}\FunctionTok{round}\NormalTok{(}\DecValTok{100}\SpecialCharTok{*}\NormalTok{resis}\SpecialCharTok{/}\FunctionTok{sum}\NormalTok{(resis),}\DecValTok{2}\NormalTok{),}\StringTok{"\%"}\NormalTok{) }
\FunctionTok{pie}\NormalTok{(resis,}\AttributeTok{col=}\FunctionTok{c}\NormalTok{(}\StringTok{"green4"}\NormalTok{,}\StringTok{"blue"}\NormalTok{,}\StringTok{"white"}\NormalTok{,}\StringTok{"black"}\NormalTok{),}\AttributeTok{density=}\FunctionTok{c}\NormalTok{(}\DecValTok{100}\NormalTok{,}\DecValTok{100}\NormalTok{,}\DecValTok{80}\NormalTok{,}\DecValTok{20}\NormalTok{),}\AttributeTok{border=}\StringTok{"black"}\NormalTok{)}
\end{Highlighting}
\end{Shaded}

\begin{center}
\pandocbounded{\includegraphics[keepaspectratio]{index_files/figure-beamer/unnamed-chunk-7-1.pdf}}
\end{center}

\textbf{Figura}: Resistência à ferrugem de híbridos de milho, para a
região de Chapecó, 1987/88
\end{block}
\end{frame}

\begin{frame}
\begin{block}{Exercícios}
\phantomsection\label{exercuxedcios}
\begin{enumerate}
\tightlist
\item
  Construa uma tabela de distribuição de frequências para a variável
  ``Tipo de grão'' de um experimento de competição de híbridos de milho
  para a região de Chapecó, SC - safra: 1987/1988 (ANDRADE; OGLIARI,
  2010, p.~62) e ilustre dois gráficos indicados.
\end{enumerate}
\end{block}
\end{frame}

\begin{frame}
\begin{block}{Distribuições de frequências de variáveis quantitativas
discretas e representações gráficas}
\phantomsection\label{distribuiuxe7uxf5es-de-frequuxeancias-de-variuxe1veis-quantitativas-discretas-e-representauxe7uxf5es-gruxe1ficas}
\begin{itemize}
\tightlist
\item
  Tabelas de \textbf{distribuições de frequências} para variáveis
  quantitativas discretas são semelhantes a tabelas de distribuições de
  frequências para variáveis qualitativas.
\end{itemize}

\begin{itemize}
\tightlist
\item
  Para esse tipo de variável os gráficos mais utilizados são: bastão,
  barras e de colunas.
\end{itemize}
\end{block}
\end{frame}

\begin{frame}
\begin{block}{Exercícios}
\phantomsection\label{exercuxedcios-1}
\begin{enumerate}
\setcounter{enumi}{1}
\tightlist
\item
  Construa a tabela de distribuição de frequências para a variável
  ``Ciclo'' de um experimento de competição de híbridos de milho para a
  região de Chapecó, SC - safra: 1987/1988 (ANDRADE; OGLIARI, 2010,
  p.~62).
\end{enumerate}
\end{block}
\end{frame}

\begin{frame}
\begin{block}{Distribuições de frequências de variáveis quantitativas
contínuas e representações gráficas}
\phantomsection\label{distribuiuxe7uxf5es-de-frequuxeancias-de-variuxe1veis-quantitativas-contuxednuas-e-representauxe7uxf5es-gruxe1ficas}
\begin{itemize}
\item
  Calcular a amplitude total dos dados:
  \[A_t=nº\ do\ maior - nº\ do\ menor = x_{max}-x_{min}.\]
\item
  Não existindo um critério ``rígido'' para estabelecer o número de
  intervalos, sugere-se não utilizar menos de 5 e não mais de 15
  intervalos.
  \[k=\sqrt(n),\ \ \ \mbox{ou}\ \ \ k=\sqrt(n)-1, \ \ \ \mbox{ou} \ \ \ k=1+3,3log(n).\]
\item
  O intervalo das classes pode ser feito dividindo-se a amplitude total
  pelo número de classes: \[a_c=\frac{A_t}{k}.\]
\end{itemize}
\end{block}
\end{frame}

\begin{frame}[fragile]
Tabela: Distribuição de frequências dos rendimentos médios, em kg/ha, de
32 híbridos de milho recomendados para o Oeste catarinense, 1987/88\}

\begin{longtable}[]{@{}
  >{\centering\arraybackslash}p{(\linewidth - 12\tabcolsep) * \real{0.1429}}
  >{\centering\arraybackslash}p{(\linewidth - 12\tabcolsep) * \real{0.1429}}
  >{\centering\arraybackslash}p{(\linewidth - 12\tabcolsep) * \real{0.1429}}
  >{\centering\arraybackslash}p{(\linewidth - 12\tabcolsep) * \real{0.1429}}
  >{\centering\arraybackslash}p{(\linewidth - 12\tabcolsep) * \real{0.1429}}
  >{\centering\arraybackslash}p{(\linewidth - 12\tabcolsep) * \real{0.1429}}
  >{\centering\arraybackslash}p{(\linewidth - 12\tabcolsep) * \real{0.1429}}@{}}
\toprule\noalign{}
\begin{minipage}[b]{\linewidth}\centering
Rendimento médio(kg/ha)
\end{minipage} & \begin{minipage}[b]{\linewidth}\centering
Freq. abs.(\(n_i\))
\end{minipage} & \begin{minipage}[b]{\linewidth}\centering
Freq. acum. (\(N_i\))
\end{minipage} & \begin{minipage}[b]{\linewidth}\centering
Freq. rel. (\(f_i\))
\end{minipage} & \begin{minipage}[b]{\linewidth}\centering
Freq. rel. acu.(\(F_i\))
\end{minipage} & \begin{minipage}[b]{\linewidth}\centering
Porc. (\%)
\end{minipage} & \begin{minipage}[b]{\linewidth}\centering
Porc. acum.
\end{minipage} \\
\midrule\noalign{}
\endhead
3973 \(\vdash\) 4457 & 1 & 1 & 0,031 & 0,031 & 3,1 & 3,1 \\
4457 \(\vdash\) 4941 & 12 & 13 & 0,375 & 0,406 & 37,5 & 40,6 \\
4941 \(\vdash\) 5425 & 13 & 26 & 0,406 & 0,812 & 40,6 & 81,2 \\
5425 \(\vdash\) 5909 & 3 & 29 & 0,094 & 0,906 & 9,4 & 90,6 \\
5909 \(\vdash\) 6393 & 3 & 32 & 0,094 & 1,000 & 9,4 & 100,0 \\
TOTAL & 32 & \(-\) & 1 & \(-\) & 100 & \(-\) \\
\bottomrule\noalign{}
\end{longtable}

\begin{Shaded}
\begin{Highlighting}[]
\FunctionTok{require}\NormalTok{(fdth)}
\NormalTok{tab}\OtherTok{=}\FunctionTok{fdt}\NormalTok{(dados}\SpecialCharTok{$}\NormalTok{Rendimento,}\AttributeTok{start =} \DecValTok{3973}\NormalTok{,}\AttributeTok{end =} \DecValTok{6393}\NormalTok{,}\AttributeTok{h =} \DecValTok{484}\NormalTok{)}
\NormalTok{tab}
\end{Highlighting}
\end{Shaded}
\end{frame}

\begin{frame}
\begin{block}{Gráficos para variáveis contínuas}
\phantomsection\label{gruxe1ficos-para-variuxe1veis-contuxednuas}
A representação gráfica de dados relativos a uma variável contínua,
apresentado em uma tabela com classes de frequências, pode ser feita
através de:

\begin{itemize}
\tightlist
\item
  Histograma
\end{itemize}

\begin{itemize}
\tightlist
\item
  Polígono de frequências
\end{itemize}
\end{block}
\end{frame}

\begin{frame}[fragile]{Histograma}
\phantomsection\label{histograma}
\begin{Shaded}
\begin{Highlighting}[]
\NormalTok{s}\OtherTok{=}\FunctionTok{seq}\NormalTok{(}\FunctionTok{min}\NormalTok{(dados}\SpecialCharTok{$}\NormalTok{Rendimento),}\DecValTok{6393}\NormalTok{,}\DecValTok{484}\NormalTok{)}
\FunctionTok{hist}\NormalTok{(dados}\SpecialCharTok{$}\NormalTok{Rendimento,}\AttributeTok{breaks=}\NormalTok{s,}\AttributeTok{xaxt=}\StringTok{"n"}\NormalTok{,}\AttributeTok{ylim=}\FunctionTok{c}\NormalTok{(}\DecValTok{0}\NormalTok{,}\DecValTok{14}\NormalTok{),}\AttributeTok{border=}\StringTok{"black"}\NormalTok{,}\AttributeTok{xlab=}\StringTok{"Rendimento médio (kg/ha)"}\NormalTok{,}\AttributeTok{density=}\DecValTok{30}\NormalTok{, }\AttributeTok{las=}\DecValTok{1}\NormalTok{,}\AttributeTok{ylab=}\StringTok{"Densidade de frequências"}\NormalTok{,}\AttributeTok{col=}\StringTok{"green4"}\NormalTok{,}\AttributeTok{main=}\StringTok{""}\NormalTok{)}
\FunctionTok{axis}\NormalTok{(}\DecValTok{1}\NormalTok{,s,}\AttributeTok{line=}\SpecialCharTok{{-}}\NormalTok{.}\DecValTok{6}\NormalTok{)}
\FunctionTok{abline}\NormalTok{(}\AttributeTok{h=}\DecValTok{0}\NormalTok{,}\AttributeTok{cex=}\DecValTok{4}\NormalTok{)}
\end{Highlighting}
\end{Shaded}

\begin{center}
\pandocbounded{\includegraphics[keepaspectratio]{index_files/figure-beamer/unnamed-chunk-9-1.pdf}}
\end{center}
\end{frame}

\begin{frame}[fragile]{Polígono de frequências}
\phantomsection\label{poluxedgono-de-frequuxeancias}
\begin{Shaded}
\begin{Highlighting}[]
\NormalTok{tab}\OtherTok{=}\FunctionTok{hist}\NormalTok{(dados}\SpecialCharTok{$}\NormalTok{Rendimento,}\AttributeTok{breaks=}\NormalTok{s,}\AttributeTok{plot=}\NormalTok{F)}
\NormalTok{pontosmedios}\OtherTok{=}\FunctionTok{c}\NormalTok{(}\DecValTok{3731}\NormalTok{,tab}\SpecialCharTok{$}\NormalTok{mids,}\DecValTok{6635}\NormalTok{)}
\NormalTok{fr}\OtherTok{=}\FunctionTok{c}\NormalTok{(}\DecValTok{0}\NormalTok{,tab}\SpecialCharTok{$}\NormalTok{counts,}\DecValTok{0}\NormalTok{)}
\FunctionTok{plot}\NormalTok{(pontosmedios,fr,}\AttributeTok{type=}\StringTok{"l"}\NormalTok{,}\AttributeTok{lwd=}\DecValTok{2}\NormalTok{,}\AttributeTok{ylim=}\FunctionTok{c}\NormalTok{(}\DecValTok{0}\NormalTok{,}\DecValTok{14}\NormalTok{),}\AttributeTok{xlab=}\StringTok{"Rendimento médio (kg/ha)"}\NormalTok{,}\AttributeTok{ylab=}\StringTok{"Densidade de frequência"}\NormalTok{,}\AttributeTok{las=}\DecValTok{1}\NormalTok{,}\AttributeTok{bty=}\StringTok{"l"}\NormalTok{,}\AttributeTok{xaxt=}\StringTok{"n"}\NormalTok{,}\AttributeTok{axes=}\NormalTok{F)}
\FunctionTok{axis}\NormalTok{(}\DecValTok{1}\NormalTok{,pontosmedios,}\AttributeTok{line=}\SpecialCharTok{{-}}\NormalTok{.}\DecValTok{6}\NormalTok{)}
\FunctionTok{axis}\NormalTok{(}\DecValTok{2}\NormalTok{,tab}\SpecialCharTok{$}\NormalTok{couts,}\AttributeTok{las=}\DecValTok{1}\NormalTok{)}
\FunctionTok{abline}\NormalTok{(}\AttributeTok{h=}\DecValTok{0}\NormalTok{)}
\end{Highlighting}
\end{Shaded}

\begin{center}
\pandocbounded{\includegraphics[keepaspectratio]{index_files/figure-beamer/unnamed-chunk-10-1.pdf}}
\end{center}
\end{frame}

\begin{frame}{Exercícios}
\phantomsection\label{exercuxedcios-2}
\begin{enumerate}
\setcounter{enumi}{2}
\tightlist
\item
  Construa a tabela de distribuição de frequências para a variável
  ``Altura de plantas'' de um experimento de competição de híbridos de
  milho para a região de Chapecó, SC - safra: 1987/1988 (ANDRADE;
  OGLIARI, 2010, p.~62) e faça os gráficos indicados. Utilizar 6
  classes.
\end{enumerate}
\end{frame}




\end{document}
